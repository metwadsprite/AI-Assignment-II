\documentclass[a4paper, 12pt]{article}
\usepackage[utf8]{inputenc}

\title{A.I. Assignment II - Documentation}
\author{Vlad Vasilescu \\ CEN 2.3B}
\date{May 2019}

\usepackage{natbib}
\usepackage{graphicx}

\begin{document}

\maketitle

\section*{Problem 1 - Water Jug Problem}
Given a number of jugs without any measuring markings of specified quantities and an infinite water
source, find a way to get to the asked for quantities in each jug.

\subsection*{The "WaterJugProblem" class:}
Implements the "Problem" class. Apart from the two constructor arguments of it's parent class 
(initial state and goal state) it also requires a third, the maximum quantity for each jug.

\subsection*{Possible actions:}
\begin{itemize}
    \item Fill a jug.
    \item Pour out the water from a jug.
    \item Pour water from one jug to another.
\end{itemize}

\subsection*{Heuristic:}
The sum of differnces between the goal and current quantity for each jug.

\subsection*{Conclusion:}
Even with a simple heuristic the informed search performs much better than the uninformed one
(roughly 100 times faster on the given example).

\section*{Problem 2 - N-Puzzle Problem}
Given a board of NxN size with $N^{2} - 1$ numbered tiles and a missing one, find a way to
get to the given arrangement of the tiles.

\subsection*{The "NPuzzleProblem" class:}
Implements the "Problem" class. It's constructor also requires a third argument, the board size.

\subsection*{Possible actions:}
\begin{itemize}
    \item Swap the empty tile with the one above it. (Not possible if the empty tile is on the
    first row)
    \item Swap the empty tile with the one below it. (Not possible if the empty tile is on the 
    last row)
    \item Swap the empty tile with the one to it's left. (Not possible if the empty tile is on
    the first column)
    \item Swap the empty tile with the one to it's right. (Not possible if the empty tile is on
    the last column)
\end{itemize}

\subsection*{Heuristics:}
\paragraph{1.}
Number of tiles not on their row plus number of tiles not on their column.

\paragraph{2.}
$h = P + N * S$ where:
\begin{itemize}
    \item P is manhattan distance of each tile to it's proper position.
    \item N is the number of lines/columns on the board.
    \item S is a score calculated by iterating through each line and adding 2
    for every tile that is not followed by it's proper successor and 0 otherwise.
    The score starts at one.
\end{itemize}
This heuristic is based upon "Nilsson's Sequence Score".

\subsection*{Conclusion:}
While the first heuristic is much simpler, it is also faster (approx. 5-10 times)
than the second. Even so the difference between using any of the heuristics and an uninformed
search is much greater. I haven't included the uninfromed search test for this problem
because it would increase the run time from under one second to a good couple of seconds.\\
Another thing we can observe is that while both the heuristics lead to the same result
(the goal state) they take different action paths.

\section*{Final note:}
The testing methodology used is measuring the time right before and after each search and
calculating the difference.

\end{document}
